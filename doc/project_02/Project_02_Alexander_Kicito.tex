\documentclass[12pt]{article}
\usepackage[english]{babel}
\usepackage{natbib}
\usepackage{url}
\usepackage[utf8x]{inputenc}
\usepackage{amsmath}
\usepackage{graphicx}
\usepackage{subfig}
\graphicspath{{../images/}}
\usepackage{parskip}
\usepackage{fancyhdr}
\usepackage{vmargin}
\usepackage{algorithm}
\usepackage[noend]{algpseudocode}
\newcommand{\var}{\texttt}
\usepackage[toc,page]{appendix}

\setmarginsrb{3 cm}{2.5 cm}{3 cm}{2.5 cm}{1 cm}{1.5 cm}{1 cm}{1.5 cm}

\title{Heuristics \& Approximations Algorithms}								% Title
\author{Alexander \& Narongrit}								% Author
\date{27 May 2019}											% Date

\makeatletter
\let\thetitle\@title
\let\theauthor\@author
\let\thedate\@date
\makeatother

\pagestyle{fancy}
\fancyhf{}
\rhead{\theauthor}
\lhead{\thetitle}
\cfoot{\thepage}

\makeatletter
\let\OldStatex\Statex
\renewcommand{\Statex}[1][3]{%
  \setlength\@tempdima{\algorithmicindent}%
  \OldStatex\hskip\dimexpr#1\@tempdima\relax}
\makeatother

\begin{document}

%%%%%%%%%%%%%%%%%%%%%%%%%%%%%%%%%%%%%%%%%%%%%%%%%%%%%%%%%%%%%%%%%%%%%%%%%%%%%%%%%%%%%%%%%

\begin{titlepage}
    \centering
    \vspace*{0.5 cm}
    \includegraphics[scale = 0.5]{SDU_logo.png}\\[1.0 cm]	% University Logo
    \textsc{\LARGE University of Southern Denmark}\\[2.0 cm]	% University Name
    \textsc{\Large DM852}\\[0.5 cm]				% Course Code
    \rule{\linewidth}{0.2 mm} \\[0.4 cm]
    { \huge \bfseries \thetitle}\\
    \rule{\linewidth}{0.2 mm} \\[1.5 cm]
    
    \begin{minipage}{0.4\textwidth}
        \begin{flushleft} \large
            \emph{Submitted To:}\\
            Marco Chiarandini\\
            Lene Monrad Favrholdt \\
            IMADA \\
            Mathematics \& Computer Science Department \\
        \end{flushleft}
    \end{minipage}~
    \begin{minipage}{0.4\textwidth}
        
        \begin{flushright} \large
            \emph{Submitted By :} \\
            Alexander Lerche Falk\\
            Narongrit Unwerawattana\\
            Spring - Master of Computer Science\\
        \end{flushright}
        
    \end{minipage}\\[2 cm]
    
    
    
    
    
    
    
\end{titlepage}

%%%%%%%%%%%%%%%%%%%%%%%%%%%%%%%%%%%%%%%%%%%%%%%%%%%%%%%%%%%%%%%%%%%%%%%%%%%%%%%%%%%%%%%%%

\tableofcontents
\pagebreak

%%%%%%%%%%%%%%%%%%%%%%%%%%%%%%%%%%%%%%%%%%%%%%%%%%%%%%%%%%%%%%%%%%%%%%%%%%%%%%%%%%%%%%%%%

\section{Introduction}

This project shows metaheuristic algorithms, resolving the Capacitated Vehicle Routing Problem (CVRP). The difference between metaheuristics and heuristics is in 
the solution part. For heuristics, you are trying your best to find a solution, even though it is not optimal. The algorithm is adapted to the problem
in such a greedy approach, it can get stuck in a local optimum. This is fine since it is the idea of heuristics: "just solve the problem". 
\newline
Metaheuristics are less greedy and tends to be more problem independent. They accept temporary solutions and allow "bad" steps as an attempt to get
a better solution. You have a local- and global optimum to keep track of the best solution found. You can say metaheuristics are exploting
heuristics to avoid getting trapped. 
\newline
In this metaheuristics implementation project, we have chosen to implement two algorithms for CVRP: Simulated Annealing (SA) and Ant Colony Optimization (ACO). 
The SA algorithm is the one we have uploaded for electronic submisison at http://valkyria.imada.sdu.dk/D0App/. The ACO algorithm is implemented but does not perform well.
We will compare the algorithms with our previous heuristic / local search project and lastly, show our results of computation for our two algorithms.

\hspace{1 cm}--- Alexander \& Narongrit

\newpage

\section{Simulated Annealing}


The Simulated Annealing (SA) algorithm is inspired from the annealing process in metal, where you are altering the physical state of the metal 
by heating and cooling it. The inspiration can be used in computer science as well. We can use the algorithm on CVRP by starting off by generating a solution 
to the problem and not "care" about the initial solutions. The better steps we are taking, and better solutions we are finding, the more careful we are going to be in finding 
a solution. In the beginning we allow random and bad steps to be performed, while later, we make better calculations.
\newline
We have developed an algorithm to solve CVRP using Simulated Annealing technique by relating CVRP with existed problems i.e. Bin packing problem and Knapsack problem. Using this approach, we can guarantee maximum amount of vehicles necessary required with approximation analysis on bin packing hueristic algorithm. In this demonstration, we use well studied  algorithm First-Fit-Decreasing which is guaranteed to use no more than $\frac{11}{9}\textsc{OPT} + 1$ number of required vehicles\cite{FFD}. With combination of simple Traveling Salesman Problem's local search algorithm, 2-opt, we are able create the initial feasible solution as described in Algorithm~\ref{alg:alg_ffd}. Next step of Simulated Annealing involved creating new feasible solution by altering current solution. Which in CVRP case, is to perform points exchange action between routes. In order to allow as much possibility of exchanges as it could without breaking capacity constrain, we choose simple algorithm to find exchanges possibility based on Knapsack problem~\ref{alg:alg_knapsackcom}. Our implementation of acceptance criterion is based on Metropolis condition which accept new solution by using probability $p = \exp[−\frac{new\_cost − best\_cost}{temp}]$, this leads to $p = 1$ when $new\_cost < best\_cost$ otherwise $p = [0,1)$. We use geometric cooling scheme in order to update temperature every iteration the algorithm accepts new solution by multiply temperature variables with $alpha$ where  $alpha=(0,1)$. This makes $p$ have lower probability to accept worsen solution after later iteration. All algorithms is described as following.

\begin{algorithm}[!h]
    \small
    \caption{Simulated annealing with First-Fit-Decreasing and Knapsack combination}
    \label{alg:alg_sa}
    
    \begin{algorithmic}[1]
        \Require{$customers$: List of customer coordinate with it's capacity}
        \Require{$max\_cap$: Maximum capacity for each route}
        \Require{$depot$: Depot point}
        
        \Require{$is\_accept(best\_cost, cost, temp)$: an acceptance criterion function returns boolean based on Metopolis condition}
        
        \Function{simulated\_annealing\_ffd}{{$temp, alpha, no\_improve\_limits$}}
        \State $routes \gets \textsc{cvrp\_first\_fit\_decreasing}(customers,depot)$
        \State $best\_cost \gets$ total cost of $routes$
        \State $best\_routes \gets routes$
        \State $current\_cost \gets best\_cost$
        \State $current\_routes \gets routes$
        \State $curr\_no\_improve\_trying \gets 0$
        \While{True}
        \If{$curr\_no\_improve\_trying == no\_improve\_limits$}
        break
        \EndIf
        \State $new\_routes \gets \textsc{neighbors\_selection}(current\_routes)$
        
        \State $new\_cost \gets$ cost of $new\_routes$
        
        \If{$is\_accept(best\_cost, new\_cost, temp)$}
        \State $temp \gets temp * alpha$
        \State $curr\_no\_improve\_trying \gets 0$
        \State update $current\_routes$ with new routes
        \If{$new\_cost < best\_cost$}
        \State update best\_cost and best\_routes with new routes
        \EndIf
        \Else
        \State $curr\_no\_improve\_trying \gets curr\_no\_improve\_trying + 1$
        \EndIf
        \EndWhile
        \Return $best\_routes$
        \EndFunction
    \end{algorithmic}
\end{algorithm}

\begin{algorithm}[!h]
    \small
    \caption{First-Fit-Decreasing for CVRP}
    \label{alg:alg_ffd}
    
    \begin{algorithmic}[1]
        \Require{$customers$: List of customer coordinate with it's capacity}
        \Require{$tsp\_local\_search(route)$: performs traveling salesman local search and returns improved route}
        \Require{$max\_cap$: Maximum capacity for each route}
        \Require{$depot$: Depot point}
        \Require{$route\_cost(route)$: returns capacity of route}
        
        \Ensure{$routes$: solution routes}
        \Function{cvrp\_first\_fit\_decreasing}{{$customers$}}
        \State $sorted\_capacity\_points \gets$ sort customer points by it's capacity
        \State $routes \gets []$
        \State add new array with $depot$ to $routes$ to create new route
        \ForAll{$point$ in $sorted\_capacity\_points$}
        \State $is\_fit \gets False$
        \ForAll{$route$ in $routes$}
        \If{$route\_cost(route) + point.capacity \leq max\_cap$}
        \State add $point$ to $route$
        \State $is\_fit \gets True$
        \State break
        \EndIf
        \EndFor
        
        \If{$!is\_fit$}
        \State add new route with $depot$ and $point$ to $routes$
        \EndIf
        \EndFor
        
        \ForAll{$route$ in $routes$}
        \State $route \gets tsp\_local\_search(route)$
        \EndFor
        \Return $routes$
        \EndFunction
    \end{algorithmic}
\end{algorithm}

\begin{algorithm}[!h]
    \small
    \caption{Neighbours selection for SA}
    \label{alg:alg_ffd}
    \begin{algorithmic}[1]
        \Require{$routes$: list of routes in solution contains customer coordinate and it's capacity required}
        \Require{$max\_cap$: Maximum capacity for each route}
        
        \Require{$route\_cap(route)$: returns capacity of $route$}
        \Require{$tsp\_local\_search(route)$: performs traveling salesman local search and returns improved route}
        
        \Ensure{$new\_routes$: new solution routes}
        
        \Function{neighbor\_selection}{{$routes$}}
        
        \Repeat
        \State $origin\_route\_id, target\_route\_id \gets$ randomly pick 2 distinct index of $routes$
        \State $origin\_route\_point \gets$ randomly pick element from $route[origin\_route\_id]$
        
        % \State $upper\_bound, lower\_bound \gets$ maximum and minimum capacity such that solution still feasible after exchange action
        
        \State $upper\_bound \gets max\_cap - (route\_cap(routes[origin\_route\_id] - origin\_route\_point.capacity))$
        
        \State $lower\_bound \gets route\_cap(routes[target\_route\_id]) - (max\_cap - origin\_route\_point.capacity)$
        
        \State $target\_neighbor\_set \gets \textsc{knapsack\_combination}($
            \Statex $route[target\_route\_id], lower\_bound, upper\_bound)$
        
        \Until{$len(target\_neighbor\_set) > 0$}
        
        \State $target\_neighbor \gets$ randomly pick an element from $target\_neighbor\_set$
        
        \State $new\_route\_origin, new\_route\_target \gets$ exchange neighbors between two route and assign to new variables
        
        \State $new\_route\_origin \gets tsp\_local\_search(new\_route\_origin)$
        \State $new\_route\_target \gets tsp\_local\_search(new\_route\_target)$
        
        \State $new\_routes \gets$ $routes$ where $route[origin\_route\_id]$ and $route[target\_route\_id]$ replaced by new routes
        \State \Return $new\_routes$
        \EndFunction
        
    \end{algorithmic}
\end{algorithm}


\begin{algorithm}[!h]
    \small
    \caption{Knapsack Combination}
    \label{alg:alg_knapsackcom}
    
    \begin{algorithmic}[1]
        \Require{$customers$: List of customer coordinate with it's capacity}
        \Require{$max\_cap$: Maximum capacity for each route}
        \Require{$depot$: Depot point}
        \Require{$max\_points\_amt$: Number of maximum combination points}
        \Require{$route\_cap(route)$: returns capacity of $route$}
        
        \Ensure{$combination\_set$: set of candidates combination of points in $route$}
        \Function{knapsack\_combination}{$route, lower\_bound, upper\_bound$}
        
        \State $points \gets$ randomly pick $max\_points\_amt$ points from $route$ but $depot$
        \State $combination\_set \gets$ []
        
        \Function{knapsack\_combination\_recur}{{$current\_point\_id, current\_capacity, knapsack$}}
        \If{$current\_point\_id == len(points)$}
        \If{$route\_cap(knapsack) > lower\_bound$}
        \State add knapsack to $combination\_set$
        \EndIf
        \EndIf
        \State $knapsack\_combination\_recur($
            \Statex $current\_point\_id + 1, current\_capacity, knapsack)$
        
        \If{$current\_cap + points[current\_point\_id] \leq max\_cap$}
        \State add $points[current\_point\_id]$ to $knapsack$
        \State $current\_cap \gets points[current\_point\_id].capacity$
        \State $knapsack\_combination\_recur($
            \Statex $current\_point\_id + 1, current\_capacity, knapsack)$
        
        \EndIf
        \EndFunction
        \State $knapsack\_combination(0,0,[])$
        \State \Return $combination\_set$
        \EndFunction
    \end{algorithmic}
\end{algorithm}

\newpage
\subsection{Complexity and Analysis}

By relating CVRP to other existed problem, we are able to come up with a simple algorithm and use existed technique, Simulated Annealing, to improve the solution. We first analyzing our algorithm to construct our initial routes. Using $\textsc{First-Fit-Decreasing}$ to assign group customer points for a vehicle only have runtime complexity of $O(nlogn)$ but in order to decide customer point sequences for each route, we rely on local search algorithm for traveling salesman problem which dominates runtime complexity of our bin packing algorithm. Using simple 2-opt local search algorithm arrange customer point sequences for each vehicle route yields runtime complexity of $O(n^2)$.

Another key task of Simulated Annealing is the neighbor selection procedure. Our attempt is to find largest possibility of exchanging customer point between routes. The problem is risen when given one customer point chosen by uniform distribution probability, how can we decide other candidate to make exchange action and such candidate should not leads to infeasible solution. Introducing a simple algorithm to find possibility of packing customer point into one set based on brute force algorithm for Knapsack problem, which explore for every customer points either to include in neighbor exchange candidate or not, allows us to find a list of feasible candidates to make exchange between routes. But tradeoff of this algorithm is the runtime complexity, which is $O(2^n)$ where $n$ is number of customer points in candidate route. In order to keep runtime of each Simulated Annealing iteration small, we introduce a user-defined variable $max\_points\_amt$ which an algorithm will use to choose number of customer points in candidate route and proceed the algorithm. This indeed reduces the number of possibility on selecting exchange neighbors candidates but in an instance where it consist of many small capacity point, without limiting such customer point might leads to very long runtime for each neighbor selection process. We conclude that for each iteration of our Simulated Annealing the runtime complexity is in $O(2^n)$ where $n$ is number of maximum points allowed to be exchange.

\begin{table}[!h]
    \footnotesize

    \begin{center}
    \begin{tabular}{|l|r|r|r|r|l|r|r|r|r|}
    \hline
    \multicolumn{ 1}{|c|}{\textbf{instance}} & \multicolumn{ 2}{c|}{\textbf{Cluster}} & \multicolumn{ 2}{c|}{\textbf{SA}} & \multicolumn{ 1}{c|}{\textbf{instance}} & \multicolumn{ 2}{c|}{\textbf{CHH}} & \multicolumn{ 2}{c|}{\textbf{SA}} \\ \cline{ 2- 5}\cline{ 7- 10}
    \multicolumn{ 1}{|c|}{} & \multicolumn{1}{c|}{\textbf{k}} & \multicolumn{1}{c|}{\textbf{cost}} & \multicolumn{1}{c|}{\textbf{k}} & \multicolumn{1}{l|}{cost} & \multicolumn{ 1}{c|}{} & \multicolumn{1}{c|}{\textbf{k}} & \multicolumn{1}{c|}{\textbf{cost}} & \multicolumn{1}{c|}{\textbf{k}} & \multicolumn{1}{l|}{cost} \\ \hline
    A-n32-k05 & 5 & 867 & 5 & 980 & CMT01 & 6 & 604 & 5 & 553 \\ \hline
    A-n33-k05 & 5 & 743 & 5 & 661 & CMT02 & 11 & 920 & 10 & 894 \\ \hline
    A-n33-k06 & 6 & 766 & 6 & 816 & CMT03 & 8 & 985 & 8 & 987 \\ \hline
    A-n34-k05 & 6 & 889 & 5 & 778 & CMT04 & 12 & 1196 & 12 & 1568 \\ \hline
    A-n36-k05 & 5 & 862 & 5 & 807 & CMT05 & 17 & 1496 & 17 & 3406 \\ \hline
    A-n37-k05 & 5 & 741 & 5 & 669 & CMT11 & 8 & 1111 & 7 & 1517 \\ \hline
    A-n37-k06 & 7 & 1112 & 6 & 949 & CMT12 & 10 & 909 & 10 & 1180 \\ \hline
    A-n38-k05 & 6 & 822 & 5 & 730 & Golden\_01 & 9 & 6055 & 9 & 8796 \\ \hline
    A-n39-k05 & 5 & 883 & 5 & 827 & Golden\_02 & 10 & 9121 & 10 & 14227 \\ \hline
    A-n39-k06 & 6 & 887 & 6 & 999 & Golden\_03 & 9 & 12144 & 9 & 17920 \\ \hline
    A-n44-k06 & 6 & 1029 & 6 & 939 & Golden\_04 & 10 & 15644 & 10 & 24338 \\ \hline
    A-n45-k06 & 7 & 1014 & 6 & 978 & Golden\_05 & 5 & 7330 & 5 & 10954 \\ \hline
    A-n45-k07 & 7 & 1250 & 7 & 1161 & Golden\_06 & 7 & 9698 & 7 & 14509 \\ \hline
    A-n46-k07 & 7 & 997 & 7 & 1039 & Golden\_07 & 9 & 11655 & 9 & 17513 \\ \hline
    A-n48-k07 & 7 & 1232 & 7 & 1128 & Golden\_08 & 10 & 12835 & 10 & 19443 \\ \hline
    A-n53-k07 & 8 & 1142 & 7 & 1054 & Golden\_09 & 15 & 590 & 14 & 653 \\ \hline
    A-n54-k07 & 8 & 1258 & 7 & 1217 & Golden\_10 & 16 & 763 & 16 & 843 \\ \hline
    A-n55-k09 & 9 & 1158 & 9 & 1300 & Golden\_11 & 18 & 963 & 18 & 1090 \\ \hline
    A-n60-k09 & 9 & 1412 & 9 & 1394 & Golden\_12 & 20 & 1192 & 19 & 1199 \\ \hline
    A-n61-k09 & 11 & 1300 & 9 & 1111 & Golden\_13 & 28 & 972 & 26 & 1175 \\ \hline
    A-n62-k08 & 8 & 1379 & 8 & 1363 & Golden\_14 & 31 & 1194 & 30 & 1471 \\ \hline
    A-n63-k09 & 10 & 1764 & 9 & 1676 & Golden\_15 & 35 & 1501 & 33 & 1743 \\ \hline
    A-n63-k10 & 11 & 1504 & 10 & 1442 & Golden\_16 & 39 & 1880 & 37 & 2092 \\ \hline
    A-n64-k09 & 9 & 1530 & 9 & 1501 & Golden\_17 & 22 & 774 & 23 & 1289 \\ \hline
    A-n65-k09 & 10 & 1274 & 9 & 1260 & Golden\_18 & 28 & 1108 & 29 & 2350 \\ \hline
    A-n69-k09 & 10 & 1297 & 9 & 1266 & Golden\_19 & 34 & 1578 & 35 & 3279 \\ \hline
    A-n80-k10 & 10 & 1903 & 10 & 1920 & Golden\_20 & 39 & 2084 & 40 & 4309 \\ \hline
    \end{tabular}
    \end{center}
    \caption{Result comparing between Heuristic algorithm and Metahuristic(SA with $max\_points\_amt = 3, temp = 1000, alpha=0.95$) running with time limit 240 seconds on an Intel(R) Core(TM) i7-2600 CPU @ 3.40GHz with 4 GB allocated RAM running Ubuntu 16.04.}
    \label{table:cost}
    \end{table}
    

\newpage
\section{Ant Colony Optimization}
The Ant Colony Optimization (ACO) algorithm comes from the evolutionary algorithms, where computer science meets nature. In this algorithm, 
we have led us to be inspired by how ants find food in the nature. They are starting by spreading out in some area, randomly. When they seem to find trails, 
which could indicate to be good trails to find food, they leave pheromone behind them. The pheromone is used by other ants to determine their probability of taking 
a trail. If they are standing in a cross-section and they have to choose, they are taking the path with the highest pheromone. 
At some point in their exploration to find the best trail to obtain food, all the ants are using the same trails to get food and get back safe home. 
\newline
We can apply the logic of the ants in the CVRP as well. By letting the instance \- the space of which the requests are "plotted" \- be the area to find a solution, 
we can start by sending one "ant" out to a point. From this point, we are going to calculate every probability of moving to the next point \- the one with the highest "score". 
We can calulate the probability of moving by the following: 
\newline
First we establish the initial pheromone levels from all the points to every counter other point: \\
\[
    M = \begin{bmatrix}
        0     & 0.50  & 0.20 \\[0.3em]
        0.125 & 0     & 0.60 \\[0.3em]
        1.20  & 0.355 & 0
    \end{bmatrix}
\]
\begin{figure}[H]
    \caption{Illustration of the ACO algorithm, showing how it can be used, and how it process next moves}
    \centering
    \includegraphics[width=0.9\textwidth]{ACO_Whiteboard.jpg}
    \label{fig:acowhiteboard}
    
\end{figure}



\section{\\ BOXPLOT FIGURES}

\newpage

\bibliographystyle{plain}
\bibliography{references}

\begin{appendices}


\end{appendices}
\end{document}